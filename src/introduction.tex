\addchap{Introduction}

La composition des mathématiques est l’une des fonctionnalités phares de TeX, donc également de LaTeX. De nombreux outils et d’innombrables \emph{packages} sont disponibles pour composer n’importe quelle mise en forme mathématique, aussi compliquée soit elle, tout en offrant un contrôle très fin sur la composition. Une des difficultés est de connaître les outils existants, et de savoir lesquels utiliser à bon escient.

Le but de ce tutoriel est de présenter une palette d'outils adaptés à des situations de composition courantes, et de donner quelques règles de bases pour bien les utiliser. Tout cela dans le but de composer des maths plus facilement.

Pour cela, nous allons d’abord faire quelques rappels sur le mode mathématique, et voir comment bien l’utiliser. Puis, nous allons nous intéresser à quelques environnements qui permettent de répondre à la majorité des besoins, avant de créer des commandes pour nous faciliter l’écriture. Nous allons finalement nous intéresser aux réglages des paramètres du mode mathématique et voir comment changer le comportement de TeX.

\begin{Information}
\begin{description}
	\item[Prérequis] : 
		\begin{itemize}
			\item Connaissances de base en LaTeX ;
			\item Connaissances des commandes de base du mode mathématique.	
		\end{itemize}
	\item[Prérequis optionnel] : Connaissances de quelques primitives de TeX.
	\item[Objectifs] : 
		\begin{itemize}
			\item Présenter les règles basiques de composition des mathématiques.
			\item Présenter des outils pour une composition efficace des mathématiques.
		\end{itemize}
\end{description}
\end{Information}
